\documentclass {article}
\usepackage[margin=2.5cm]{geometry}
\usepackage {polski}
\usepackage {float}
\usepackage {graphicx}
\usepackage{indentfirst}
\usepackage[utf8]{inputenc}

\author {Michał Cybulski, Piotr Grzegorski}
\title {PORR -- Projekt\\Etap 1.}
\date {30 listopada 2015r.}
\begin {document}

\maketitle

\section {Zadanie}

Wyznaczanie najkrótszej ścieżki w grafie metodą aukcyjną, porównanie z algorytmem Dijkstry.

Zdecydowano o realizacji zadania przy pomocy technologii OpenMP.

\section{Implementacja}

Program został napisany w języku C.

\subsection{Argumenty wywołania}

V E MW poziom logowania

\subsection{Przebieg działania programu}

Generuj --> wywołaj Dijkstrę --> jeśli graf niespojny odrzucamy wynik, jesli spojny wywołaj aukcyjny --> end

\subsection{Zrównoleglenie algorytmu Dijkstry}

Każdy wątek sprawdza odległość dla podległych sobie wierzchołków.

\subsection{Zrównoleglenie algorytmu aukcyjnego}

Inaczej.

\section{Testy}

\subsection{Jakość zrównoleglenia}

Testy wywołań sekwencyjnych wykonano przez ustawienie zmiennej środowiskowej sterującej OpenMP:

\begin{verbatim}
export OMP_NUM_THREADS=1
\end{verbatim}

Zgodnie z przewidywaniami w obu wypadkach wydajność wzrosła po zrównolegleniu obliczeń. Efekty były bardziej wyraźne dla algorytmu ........................

\paragraph{Dijkstra}

%tabelka wyników

\paragraph{Aukcyjny}

%tabelka wyników


\subsection{Porównanie algorytmów}

Algorytmy były wykonywane dla tych samych grafów, grafy generowane w rozmiarach A, B i C po 100 razy.

%tabelka wyników

\section{Wnioski}



\end {document}
